\documentclass[journal,12pt,twocolumn]{IEEEtran}
%

\usepackage{setspace}
\usepackage{gensymb}
\singlespacing

\usepackage{amsmath}
\usepackage{amsthm}
\usepackage{txfonts}
\usepackage{cite}
\usepackage{enumitem}
\usepackage{mathtools}
\usepackage{listings}
    \usepackage{color}                                            %%
    \usepackage{array}                                            %%
    \usepackage{longtable}                                        %%
    \usepackage{calc}                                             %%
    \usepackage{multirow}                                         %%
    \usepackage{hhline}                                           %%
    \usepackage{ifthen}                                           %%
  %optionally (for landscape tables embedded in another document): %%
    \usepackage{lscape}     
\usepackage{multicol}
\usepackage{chngcntr}
\usepackage{tikz}
\usepackage{pgfplots}
\renewcommand\thesection{\arabic{section}}
\renewcommand\thesubsection{\thesection.\arabic{subsection}}
\renewcommand\thesubsubsection{\thesubsection.\arabic{subsubsection}}

\renewcommand\thesectiondis{\arabic{section}}
\renewcommand\thesubsectiondis{\thesectiondis.\arabic{subsection}}
\renewcommand\thesubsubsectiondis{\thesubsectiondis.\arabic{subsubsection}}

% correct bad hyphenation here
\hyphenation{op-tical net-works semi-conduc-tor}
\def\inputGnumericTable{}                                 %%

\lstset{
%language=C,
frame=single, 
breaklines=true,
columns=fullflexible
}
\begin{document}
\newtheorem{theorem}{Theorem}[section]
\newtheorem{problem}{Problem}
\newtheorem{proposition}{Proposition}[section]
\newtheorem{lemma}{Lemma}[section]
\newtheorem{corollary}[theorem]{Corollary}
\newtheorem{example}{Example}[section]
\newtheorem{definition}[problem]{Definition}
\newcommand{\BEQA}{\begin{eqnarray}}
\newcommand{\EEQA}{\end{eqnarray}}
\newcommand{\define}{\stackrel{\triangle}{=}}
\bibliographystyle{IEEEtran}
\providecommand{\mbf}{\mathbf}
\providecommand{\pr}[1]{\ensuremath{\Pr\left(#1\right)}}
\providecommand{\qfunc}[1]{\ensuremath{Q\left(#1\right)}}
\providecommand{\sbrak}[1]{\ensuremath{{}\left[#1\right]}}
\providecommand{\lsbrak}[1]{\ensuremath{{}\left[#1\right.}}
\providecommand{\rsbrak}[1]{\ensuremath{{}\left.#1\right]}}
\providecommand{\brak}[1]{\ensuremath{\left(#1\right)}}
\providecommand{\lbrak}[1]{\ensuremath{\left(#1\right.}}
\providecommand{\rbrak}[1]{\ensuremath{\left.#1\right)}}
\providecommand{\cbrak}[1]{\ensuremath{\left\{#1\right\}}}
\providecommand{\lcbrak}[1]{\ensuremath{\left\{#1\right.}}
\providecommand{\rcbrak}[1]{\ensuremath{\left.#1\right\}}}
\theoremstyle{remark}
\newtheorem{rem}{Remark}
\newcommand{\sgn}{\mathop{\mathrm{sgn}}}
\providecommand{\abs}[1]{\left\vert#1\right\vert}
\providecommand{\res}[1]{\Res\displaylimits_{#1}} 
\providecommand{\norm}[1]{\left\lVert#1\right\rVert}
\providecommand{\mtx}[1]{\mathbf{#1}}
\providecommand{\mean}[1]{E\left[ #1 \right]}
\providecommand{\fourier}{\overset{\mathcal{F}}{ \rightleftharpoons}}
\providecommand{\system}{\overset{\mathcal{H}}{ \longleftrightarrow}}
\newcommand{\solution}{\noindent \textbf{Solution: }}
\newcommand{\cosec}{\,\text{cosec}\,}
\providecommand{\dec}[2]{\ensuremath{\overset{#1}{\underset{#2}{\gtrless}}}}
\newcommand{\myvec}[1]{\ensuremath{\begin{pmatrix}#1\end{pmatrix}}}
\newcommand{\cmyvec}[1]{\ensuremath{\begin{pmatrix*}[c]#1\end{pmatrix*}}}
\newcommand{\mydet}[1]{\ensuremath{\begin{vmatrix}#1\end{vmatrix}}}
\newcommand{\proj}[2]{\textbf{proj}_{\vec{#1}}\vec{#2}}
\let\StandardTheFigure\thefigure
\let\vec\mathbf
\title{Assignment - 4}
\author{Mr. Ganesh Yadav
\\ MD/2020/712}
% make the title area
\maketitle
\newpage
%\tableofcontents
\bigskip
\renewcommand{\thefigure}{\theenumi}
\renewcommand{\thetable}{\theenumi}
%\renewcommand{\theequation}{\theenumi}
%Download all python codes 
%
%\begin{lstlisting}
%svn co https://github.com/JayatiD93/trunk/My_solution_design/codes
%\end{lstlisting}
Download all and latex-tikz codes from 
%
\begin{lstlisting}
svn co https://github.com/Ganeshyadav712/Assignment-4.git
\end{lstlisting}
%
Question taken from
\begin{lstlisting}
https://github.com/gadepall/ncert/blob/main/linalg/linear_forms/gvv_ncert_linear_forms.pdf- 2.3 a,c  
\end{lstlisting}
%
\section{Question}
Draw the graphs of the following equations
\begin{enumerate}
    \item   \label{eq:1} \myvec{1&1}\vec{x}&=4\\
    \item   \label{eq:2} \myvec{3&-1}\vec{x}&=0
\end{enumerate}
\section{Solution}
\begin{equation}
    \label{eq:1} \myvec{1&1}\vec{x}&=4
\end{equation}
let \vec{x}=\myvec{a\\0} substitute in \eqref{eq:1}
\begin{align}
  \myvec{1&1}\myvec{a\\0}&=4\\
a&=4
\end{align}
similarly let \vec{x}=\myvec{0\\b}substitute in \eqref{eq:1}
\begin{align}
\myvec{1&1}\myvec{0\\b}&=4\\
b&=4
\end{align}
intercept on X and Y axis for equation 1 can be
$$\vec{A}=\myvec{4\\0}, \vec{B}=\myvec{0\\4}$$
\begin{align}
 2)\label{eq:6} \myvec{3&-1}\vec{x}&=0   
\end{align}
\begin{enumerate}
    \item In equation (6) there is no constant thus line passes through origin,
\end{enumerate}
so
\begin{align}
\vec{P}&=\myvec{0\\0}
\end{align}
for $\vec{Q}$ point,\\
let \vec{x}=\myvec{x\\y} substitute in \eqref{eq:2}  
\begin{equation} \label{eq1}
\myvec{3&-1}\myvec{x\\y}=0
\end{equation}
\begin{align}
{x}&=1\\
\implies {y}&=3\\
\vec{Q}&=\myvec{1\\3}
\end{align}
$$\vec{P}=\myvec{0\\0}, \vec{Q}=\myvec{1\\3}$$\\
Graphs of the both equations constructed by using python as
\numberwithin{figure}{section}
\begin{figure}[h]
    \centering
    \includegraphics[width=\columnwidth]{ASSIGN 4 Ffig.png}
\caption{Graph 4}
    \label{ASSIGNMENT 4.png}
\end{figure}
\end{document}